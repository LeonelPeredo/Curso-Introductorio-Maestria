\documentclass{article}

\usepackage[utf8]{inputenc}
\usepackage{courier}
\usepackage{listings}


\title{R Clase 1}
\author{Leonel Peredo}

\begin{document}

\maketitle

\begin{enumerate}
\item 
Escriba: \\
a) Una función que convierta una temperatura en grados Celsius a grados Fahrenheit \\
b) Otro programa que se haga la operación inversa (de Fahrenheit a Celsius) \\
c) Verifique que aplicar una función seguida de la otra devuelven el valor original. 

\item  Cree una función que dado un cuadrado, a partir de la variable \texttt{base}, calcule su perímetro y su área.

\item Construya una función \texttt{es\_cuadrado\_perfecto} que devuelva un \textbf{booleano} según el input \texttt{x} es o no un cuadrado perfecto (existe $k$ tal que $k^2$)

\item Arme una función que dado un vector numérico \texttt{x} devuelve un vector con la media muestral y, la mediana. Use el siguiente modelo como base:

\begin{lstlisting}[language=R]
medias = function(x) {
    return(c(mean(x),median(x)))
}
\end{lstlisting}

Una vez definida la función medias ejecute la siguente línea:

\begin{lstlisting}[language=R]
replicaciones = replicate(10000, medias(rnorm(100)))
\end{lstlisting}

Extraiga de la matriz \texttt{replicaciones} las medias y medianas y asígnelas a los vectores  \texttt{las\_medias} y  \texttt{las\_medianas} respectivamente. Luego ejecute el siguiente código para graficar:

\begin{lstlisting}[language=R]
colores = c("skyblue3", "orange")
densidades = c(50, 30)
# las_medias = ...
hist(las_medias, density=densidades[1], col = colores[1])
# las_medianas = ...
hist(las_medianas, density=densidades[2], col = colores[2]
    , add=TRUE)
legend("topright", legend=c("media", "mediana"), col = colores, 
    density=densidades, fill = colores)
\end{lstlisting}

\item Una empresa regala una heladera al azar entre $N$ personas. Repite el proceso $n$ veces. ¿Cuál es la probabilidad de que una persona reciba dos heladeras? \\
a) Primero escriba una función que simule una sola realización del sorteo. Usá el siguiente modelo:

\begin{lstlisting}[language=R]
una_heladera = function(N, n){
    # ...
    return(dos_heladeras_bool)
}
\end{lstlisting}

b) Use el siguiente código para obtener un vector con los resultados de correr la función anterior $k$ veces.

\begin{lstlisting}[language=R]
k = 1000
N = 40000000
n = 15000
reps = replicate(k, una_heladera(N, n))
\end{lstlisting}

c) Calcule la media muestral de \texttt{reps}, ¿qué representa?.

\item Considere la siguientes tres sucesiones:
\begin{itemize}
\item $a_n = \frac{1}{\sqrt{n}} + (\frac{1}{2})^n$
\item $b_n=(-1)^n + 5$
\item $c_n = \frac{3n-8}{\&sqrt{16n^2+n}}$
\end{itemize}

a) Grafique los primeros $n=100$ términos de cada sucesión. \\
b) Decida si las sucesiones son convergentes y en tal caso agregue una línea horizontal al gráfico.

\item Considere las siguientes funciones:
\begin{itemize}
\item $l_2(x) = x^2$
\item $l_1(x) = |x|$
\item $\rho _k(x) = x^2 * I_{(|x| \leq k)} + 2k|x| - k^2 * I_{(|x| > k)}$
\end{itemize}

para $x \in \left[-10,10 \right]$, grafique $\rho _k$ con $k=5$ junto a las otras funciones con distintos colores y una leyenda.

 \item Grafique el polinomio interpolador de Lagrange en $n+1$ puntos equiespaciados en el intervalo $\left[-1,1 \right]$ , con $n=5$,$10$,$15$, para los valores generados por las siguientes funciones:

\begin{itemize}
\item $f_1(x) = \frac{1}{25x^2}$
\item $f_2(x) = \sin (\pi x)$
\end{itemize}

\end{enumerate}

\end{document}
